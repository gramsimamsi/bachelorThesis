\chapter*{Abstract (german)}
\thispagestyle{empty}
%TODO erster satz: feststellung. kannst das hinterlegen oder umformulieren? Auch im EN-Abstract!
Die wachsende Zahl von Platform-as-a-Service (PaaS) Lösungen, Cloud-Umgebungen, containerisierter Datenverarbeitung und Microservice-Architekturen bieten neue Angriffsszenarien.
Dies erhöht den Bedarf an neuen Verteidigungs-Strategien in der IT-Sicherheit von Entwicklungs- und Produktiv-Betriebsumgebungen.
Gerade Lösungen auf Basis von (Kubernetes konformer) Container-Orchestrierung sind in der Wirtschaft stark gefragt und haben einen sehr unterschiedlichen Aufbau im Kontrast zu konventionellen und etablierten Lösungen.
Dieser Umstand legt eine nähere Untersuchung dieses Teilbereichs nahe.
Ziel dieser Arbeit ist es, folgende Fragestellungen zu beantworten:

\begin{itemize}

\item Was für Sicherheits-Risiken existieren für den Anbieter und/oder Nutzer einer mandantenfähigen PaaS-Lösung, wenn jeder Mandant über eigene Entwicklungs-, Verteilungs- und Laufzeit-Umgebungen für seine Anwendungen verfügt?

\item Wie kann ein Anbieter von PaaS-Lösungen an interne und/oder externe Nutzer diese Risiken eindämmen?

\item Was spricht in diesem Kontext aus Sicht eines PaaS-Anbieters jeweils für und gegen selbst betriebene bzw. Cloud-Lösungen?

\end{itemize}

Ein weiteres Ziel ist es, Maßnahmen für die unterschiedlichen Implementierungsmöglichkeiten zu empfehlen.
Der Vergleich bestehender IT-Sicherheits-Risiken und daraus abgeleitet der priorisierte Handlungsbedarf sollen hierbei als zentraler Betrachtungswinkel dienen.
Unter den etablierten PaaS-Lösungen in unterschiedlichen Umgebungen finden sich OpenShift Container Platform als Vor-Ort-Lösung und Azure Kubernetes Service für Cloud-Umgebungen.
%ZUERST nicht mehr richtig hier. Umformulieren, auch in Englisch! Formulierung allgemein!
Um die genannten Ziele zu erreichen, grenzt diese Arbeit den Problembereich zuerst ein, indem sie sich auf die standardmäßig eingerichteten und zum Betrieb notwendigen Komponenten dieser zwei gängigen und Kubernetes-zertifizierten PaaS-Lösungen konzentriert.

Das Hauptaugenmerk liegt hierbei auf den Kubernetes-konformen Teilkomponenten, da hier die Risiken und Maßnahmen den weitreichensten Gültigkeitsbereich haben, unabhängig von der betrachteten Lösung.
Aus den Zielen werden drei gängige Angriffsszenarien hergeleitet: 

\begin{itemize}

\item Angriff von böswilligen Dritten auf die Infrastruktur ausgehend des LAN und/oder dem Internet

\item Angriff von böswilligen Dritten aus einem Container heraus, über den die Kontrolle übernommen wurde. Ein Beispiel wäre das Ausführen von Code oder Befehlen per Zugriff von außen.

\item Angriff von böswilligen Dritten sowie böswilliger oder fahrlässiger Nutzer bzw. Nutzer-Identität, was ein Kompromittierungsrisiko für diesen und/oder weitere Mandanten und deren Anwendungen darstellt.

\end{itemize}

Diese Szenarien dienen als Grundlage zur Identifikation von Angriffsvektoren, deren jeweiliges Schadenspotential evaluiert und das bestehende Risiko eingeschätzt werden.
Es werden potentielle Maßnahmen zur Risiko-Eindämmung gesucht, evaluiert und exemplarisch in der Praxis umgesetzt.
Im Anschluss wird eine Neubewertung des Risikos vorgenommen, um so beispielhaft eine zielführende Methode zum Risikomanagement zu demonstrieren.
Mithilfe der Ergebnisse werden verschiedene Implementierungs-Empfehlungen verglichen, anhand der verschiedenen Anwendungsfälle differenziert und jeweils Maßnahmen empfohlen.

\bigskip

\noindent
Schlagworte: \\
Platform-as-a-Service, PaaS, Cloud, Security, Container, Kubernetes, Docker, Risikomanagement, OpenShift Container Platform, OCP, Azure Kubernetes Service, AKS

\chapter*{Abstract (english)}
\thispagestyle{empty}

The increasing amount of Platform-as-a-Service (PaaS) solutions, cloud-hosted environments, containerized workloads and microservice architectures introduce new attack scenarios. 
This creates the need for new defense strategies in both Development and Operations. 
Especially solutions providing (Kubernetes compliant) container orchestration are identifiably different and high in business demand compared to long established solutions. 
This calls for a more detailed, focused examination. 
The thesis aims to answer the following questions:

\begin{itemize}

\item What generic security risks emerge when providing or using a multi-tenant PaaS solution,
with each tenant developing, deploying and running their own applications? 

\item How can a PaaS provider (serving internal and/or external users) mitigate those risks? 

\item  In this scope and from a PaaS provider viewpoint, how does an on-premise solution compare
to a public cloud solution? 

\end{itemize}

Another goal is to recommend security measures for different implementation use cases.
A comparison of existing IT security risks and the need for action derived from it should serve as central point of view.
Examples for widely used PaaS solutions in different environments include OpenShift Container Platform 
as an on- premise solution and Azure Kubernetes Service as a public cloud solution.
To achieve the aforementioned goals, the thesis will first limit the view on the problem to a manageable scope by
concentrating on the components enabled by default and those required for operations of these established and Kubernetes compliant solutions.
Components providing Kubernetes compliance will be the main focus, as these bear the most significance across all Kubernetes Certified solutions. 
Three common attack scenarios will be derived from the goals:

\begin{itemize}

\item Malicious third party attacking the underlying infrastructure from within the LAN and/or the
internet

\item Malicious third party attacking from inside a hijacked container, i.e. remotely executing code
or commands

\item Bad User, i.e. a negligent, hijacked or malicious developer account risking compromise of
his own and/or other applications

\end{itemize}

These scenarios serve as a foundation to identify attack vectors, evaluate their respective potential impact and estimate their risks.
Possible measures to mitigate those risks will also be explored, evaluated and exemplary put to use in practical examples.
Subsequently the risk will be re-evaluated in order to illustrate a viable, result oriented risk management method.
With the results gathered, the thesis will compare different best practice implementations for different use cases and recommend measures for each.


\bigskip

\noindent
Keywords: \\
Platform-as-a-Service, PaaS, Cloud, Security, Container, Kubernetes, Docker, risk management, OpenShift Container Platform, OCP, Azure Kubernetes Service, AKS

