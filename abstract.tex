\chapter*{Kurzfassung (deutsch)}
\thispagestyle{empty}

TODO

\chapter*{Kurzfassung (englisch)}
\thispagestyle{empty}

The increasing amount of Platform-as-a-Service (PaaS) solutions, cloud-hosted environments and
microservice architectures introduces new attack scenarios. This creates the need for new defense
strategies in both Development and Operations. Especially solutions providing (Kubernetes
conformant) container orchestration are identifiably different and in high demand compared to long
established solutions. This calls for a more detailed, focused examination. \\
This thesis aims to answer the following questions: \\

- What generic security risks emerge when providing or using a multi-tenant PaaS solution,
with each tenant developing, deploying and running their own applications? \\

- How can a PaaS provider (serving internal and/or external users) mitigate those risks? \\

-  In this scope and from a PaaS provider viewpoint, how does an on-premise solution compare
to a public cloud solution? \\

Another goal is to recommend security measures for different implementation use cases.
If achievable within the provided time frame, it will additionally try and answer this question: \\

- Does a cloud-hosted, self-managed solution (buy IaaS, provide PaaS) offer benefits in
contrast to the solutions compared above?
(In case there are sufficient differences, another set of security measures will be
recommended for this solution) \\

Examples for widely used PaaS solutions in different environments include OpenShift as an on-
premise solution, Azure Kubernetes Service as a public cloud solution. To include a cloud-hosted,
self-managed solution, OpenShift running on self-managed instances in Azure could serve as an
example. \\
To achieve these goals, the thesis will first limit the view on the problem to a manageable scope by
focusing on specific components of a few (2-3) commonly used PaaS solutions, specifically Certified
Kubernetes solutions. Components providing Kubernetes conformity will be the main focus, as these
bear the most significance across all Kubernetes Certified solutions. \\
Looking at three common attack scenarios, it will then determine vulnerabilities and attack vectors,
as well as their potential damage and rate those risks: \\

- Malicious third party attacking the underlying infrastructure from within the LAN and/or the
internet \\

- Malicious third party attacking from inside a hijacked container, i.e. remotely executing code
or commands \\

- Bad User, i.e. a negligent, hijacked or malicious developer (account) risking compromise of
his own and/or other applications \\

Possible measures to mitigate those risks will also be explored, evaluated and (if possible) put to use
in practical examples, leveraging the PaaS solutions within scope. With the results gathered, the
thesis will compare different best practice implementations for different use cases and recommend
measures for each.

\bigskip

\noindent
Schlagworte: 

