\chapter{Introduction}
With this chapter, the reader should be able to comprehend why this thesis was written, what it tries to accomplish and which topics are considered in this work.

\section{Motivation}

The increasing amount of Platform-as-a-Service (PaaS) solutions, cloud-hosted environments and
microservice architectures introduces new attack scenarios. This creates the need for new defense
strategies in both Development and Operations. Especially solutions providing (Kubernetes
conformant) container orchestration are identifiably different and in high demand compared to long
established solutions. This calls for a more detailed, focused examination. \\

\section{Objective}


This thesis aims to answer the following questions:

\begin{itemize}

\item What generic security risks emerge when providing or using a multi-tenant PaaS solution,
with each tenant developing, deploying and running their own applications? 

\item How can a PaaS provider (serving internal and/or external users) mitigate those risks? 

\item  In this scope and from a PaaS provider viewpoint, how does an on-premise solution compare
to a public cloud solution? 

\end{itemize}

Another goal is to recommend security measures for different implementation use cases.
If achievable within the provided time frame, it will additionally try and answer this question: 

\begin{itemize}

\item Does a cloud-hosted, self-managed solution (buy IaaS, provide PaaS) offer benefits in
contrast to the solutions compared above? \\
(In case there are sufficient differences, another set of security measures will be
recommended for this solution)

\end{itemize}


\section{Scope limitation}

To achieve the aforementioned goals, the view on the problem is limited to a manageable scope by
focusing on specific components of a few commonly used (Certified Kubernetes) PaaS solutions.
These components will be OpenShift Container Platform for On-Premise use cases and Azure Kubernetes Service as a public cloud solution. 
To include a cloud-hosted and self-managed solution, OpenShift running on self-managed instances in Azure serve as an
example implementation. \\
Components providing Kubernetes conformity will be the main focus, as these bear the most significance across all Kubernetes Certified solutions. \\
A look at popular tools and frameworks used in such clusters will be avoided in order to keep the scope manageable, though some might be recommended as a mitigation.\\
This thesis aims to provide insight to the risks of providing a PaaS solution and mitigations thereof. 
As such, it will look at the capabilities a potential provider has to (mis-)configure such solutions - inherent risks of the technologies themselves are only explored when measures to mitigate them are accessible from a provider standpoint. \\
In short, the goal is to improve the security of your Kubernetes cluster, not Kubernetes itself.\\
Looking at three common attack scenarios, it will then determine vulnerabilities and attack vectors,
as well as their potential damage and rate those risks:

\begin{itemize}

\item Malicious third party attacking the underlying infrastructure from within the LAN and/or the internet

\item Malicious third party attacking from inside a hijacked container, i.e. remotely executing code or commands

\item Bad user, i.e. a negligent, hijacked or malicious developer (account) risking compromise of his own and/or other applications

\end{itemize}

Possible measures to mitigate those risks will also be explored, evaluated and (if possible) put to use in the practical examples. 
With the results gathered, the thesis will compare different best practice implementations for different use cases and recommend measures for each.


TODO: Versioning Freeze! Openshift, AKS (not possible...), Docker engine, Kubernetes versions.

\chapter{Theory}
With this chapter, a reader with foundational knowledge of topics regarding Computer Science and/or Informatics should be able to grasp the specialized technologies discussed within the thesis and familiarize themselves with the definitions and terminology used throughout this thesis.

\section{Infrastructure-as-a-Service}
In \gls{iaas} environments, a consumer trusts his \gls{iaas} provider with the management and control of the infrastructure needed to deploy his applications.
The provided service ends at provisioning or processing, storage, networks and other computing resources \cite{nistcloud}
(TODO Quelle: https://nvlpubs.nist.gov/nistpubs/Legacy/SP/nistspecialpublication800-145.pdf ). 
Therefore consumers do not need to manage their own data centers or topics like system uplink availability.
As shown in Figure ~\ref{fig:servicecomparison}, consumers are responsible for the \gls{os} layer and everything above it.

(TODO: you manage <-> other manages https://www.bmc.com/blogs/saas-vs-paas-vs-iaas-whats-the-difference-and-how-to-choose/ )

citation test 1\cite{servicecomparison} \\
cit test 3~\cite{servicecomparison} \\
cit \citation{servicecomparison}

\begin{figure}[H]
\includegraphics[scale=0.4]{pictures/ServiceComparison.jpg} 
\caption{captionright\protect\footnotemark}
%% \caption{Comparison of responsibilities in different service models }
\label{fig:servicecomparison}
\end{figure}

\footnotetext{Source:  \cite{servicecomparison} }

\section{Platform-as-a-Service}

In \gls{paas} environments, a consumer trusts his \gls{paas} provider with the management and control of even more resources needed to deploy his applications beyond those covered by \gls{iaas}. 
(TODO Quelle: https://nvlpubs.nist.gov/nistpubs/Legacy/SP/nistspecialpublication800-145.pdf )
In an ideal scenario, this leads to a consumer not having to concern himself with the underlying network, hardware, servers, operating systems, storage or even common middleware like log data collection and analysis 
(TODO Quelle: https://docs.microsoft.com/en-us/azure/data-explorer/ )
 and allows him to focus on other tasks, i.e. application development. As a downside to this, a consumer might only have limited control on, among others, the software installed on the  machines provisioned. 
Although this shifts some of the responsibility burden towards the provider, the situation isn't as clear-cut as one might think. 
Figure X  (TODO: Picture link) shows middleware and runtime as provided, but there is no clear standard on what capabilities or tools are included.
A consumer might require capabilities which aren't provided or wishes to avoid provider lock-in through proprietary tools, again resulting in middleware responsibilities for the consumer. 
A consumer might also have to extensively configure the application-hosting environment for compliance or security purposes. 
Even some low-level tasks aren't completely managed, i.e. \gls{vm} reboots to apply security updates. 
(TODO Quelle: https://docs.microsoft.com/en-us/azure/aks/operator-best-practices-cluster-security?view=azuremgmtbilling-1.1.0-preview\#process-node-updates-and-reboots-using-kured )


\section{Containers, Docker}
TODO

concept of container vs. VM, upsides vs. downsides (?), docker-specific b/c highest market share
automation \& load elasticity -> need for orchestration

https://www.docker.com/resources/what-container\\
https://www.docker.com/sites/default/files/d8/2018-11/docker-containerized-and-vm-transparent-bg.png
Citations: TODO Look for books! 

\section{Container Orchestration}
TODO

idea: control system, architecture (control plane vs. application(s), kubelet, master/non-master nodes, ...), standard (can have other implementations), certification requirements, uses in general and in AKS/Openshift specifically.

\subsection{Kubernetes}

https://kubernetes.io/docs/concepts/\#overview \\
https://kubernetes.io/docs/concepts/overview/components/ \\
https://res.cloudinary.com/dukp6c7f7/image/upload/f\_auto,fl\_lossy,q\_auto/s3-ghost/2016/06/o7leok.png \\

\subsubsection{Kubernetes Objects}
TODO
higher-level idea behind objects
Pods, Services, Volumes, Namespaces

\subsubsection{Kubernetes Controllers}
TODO

higher-level idea behind controllers
ReplicaSets, Deployments, StatefulSets, DaemonSets, Jobs

\subsubsection{Using Kubernetes}
TODO

build from scratch, buy CaaS/PaaS/IaaS, cloud vs. on-prem, different scopes/features from different products

https://kubernetes.io/docs/setup/pick-right-solution/

\subsubsection{OpenShift}
TODO

\subsubsection{Azure Kubernetes Service}
TODO

\section{TODO: Others?}
TODO

\chapter{Potential attack scenarios}
TODO

\section{Defining procedures and approach}
Focus on three scenarios: attack through network, hijacked container, bad user
TODO

\section{Hijack running container}
Out of scope, b/c AppSec? But: defense measures through k8s (image scanning, pipeline)!
TODO

\section{Access kubernetes management components}
dashboard, apiserver, etcd, ..., AND kubelet! -> master + worker nodes
TODO

\section{Access cloud provider management components}
TODO

\section{Access non public-facing application components}
Out of scope, b/c AppSec? But: defense measures through k8s!
TODO

\section{Lateral movement}
Duplicate to non-public-facing? But: Inside in container != inside in network!
TODO

\section{Container breakout}
TODO

\section{Read/Manipulate host}
TODO

\section{Access host / base infrastructure from outside?}
Out of scope, b/c classic OpSec? But: included in pitch!
TODO

\section{Soft- and Hardware reconnaissance}
TODO

\section{Gather secrets, credentials}
TODO

\section{Start new container(s)}
TODO

\section{Access non-public data}
TODO

\section{Manipulate container images}
TODO

\section{Evade detection/logging}
TODO

\section{Move to cluster-external areas}
Out of scope? But: defense measures include cluster-internal things (credential reuse, cloud provider subscriptions, mixing container/non-container workloads, network restrictions)!
TODO

\section{Evade detection}
TODO

\section{Misuse of resources (Cryptojacking)}
TODO

\section{Denial of resources (DOS)}
TODO

\section{Influence other tenants}
duplicate?
TODO

\section{TODO: more?}
TODO

\chapter{PaaS solution risk analysis}
TODO

\section{Differentiating on-premise and public cloud environments}
TODO

\section{OPTIONAL: Comparison of cloud-hosted, self-managed environments}
If this is omitted, the section above will become the chapter
TODO

\chapter{Best-Practice implementations}
TODO

\section{On-premise environment}
TODO

\section{Public cloud environment}
TODO

\section{OPTIONAL: cloud-hosted, self-managed environment}
TODO

\chapter{Summary}
TODO




\chapter{Einleitung}
\section{Anführungszeichen, Zitieren, Fußnoten} 
\label{kap:beschda}
Quellen stehen in der Datei thesis.bib!\\
Da gibt es auch ein tag, um Bücher von Online-Quellen zu unterscheiden.\\

So wird zitiert\cite{booktest}. Das ist eine Quelle für Bücher. \\
Das ist eine Online-Quelle\cite{wikimoscow}.
Um kleine Definitionen zu verlinken benutze ich Fußnoten\footnote{\url{ https://www.eclipse.org/4diac/} (Abgerufen am 07.02.2019)}

So werden in Deutschland Anführungszeichen gemacht: \glqq Sehr wichtig!!\grqq{}  \\

Die Abkürzungen stehen in der Datei abbreviations.tex! \\
Das abkürzungsverzeichnis wird natürlich automatisch erstellt.\\

So eine \gls{led} ist schon toll.

\begin{figure}[H]
\includegraphics[scale=0.5]{pictures/GatewayKommunikation.png} 
\caption{Bild wird automatisch im Abbildungsverzeichnis eingefügt!}
\label{fig:gateway}
\end{figure}


\chapter{Einführung und Analyse der Anforderungen}
Die Abbildung \ref{fig:archgrob} zeigt...
\begin{figure}[H]
\includegraphics[scale=0.3]{pictures/Architektur-grob.png} 
\caption{Bild wird automatisch im Abbildungsverzeichnis eingefügt!}
\label{fig:archgrob}
\end{figure}


\section{Tabelle und Liste}

Hier gibts eine Fancy Tabelle:\\
\begin{table}[H]
\begin{tabularx}{\textwidth}{| c | X | X | c |}	\hline	
\textbf{Nr.} & \textbf{Anforderung} & \textbf{Erklärung} & \textbf{Priorisierung} \\ 
\hline	
(1a) & Einfache Installation & einfache \glqq Out-of-the-box\grqq{} Installation und intuitives Anlegen von Nutzerkonten & Must \\
\hline 
\end{tabularx}
\renewcommand{\arraystretch}{1}\\	
\caption{Anforderungen an die Sprachsteuerung} 	
\label{tab:anfsprachsteuerung}
\end{table}


Liste um alles etwas in die Länge zu strecken:
\begin{itemize}\itemsep0pt
	\item Blah
	\item Blah
\end{itemize}

Bilder:

\begin{figure}[H]
\includegraphics[scale=0.3]{pictures/4diac.png} 
\caption{Bild wird automatisch im Abbildungsverzeichnis eingefügt!}
\label{fig:4diaclogo}
\end{figure}

\begin{figure}[H]
\includegraphics[scale=1]{pictures/IEC61499Modelle.png} 
\caption{Bild wird automatisch im Abbildungsverzeichnis eingefügt!}
\label{fig:iecwiki}
\end{figure}


\chapter{Automat mit tikz}

Falls jemand bock hat einen Automaten zu basteln, hier ist eine kleine Vorlage:

%siehe http://texample.net/tikz/examples/state-machine/

\begin{figure}[H]
\begin{tikzpicture}[->,>=stealth',shorten >=1pt,auto,node distance=4.5cm,
 thick,state/.style={circle,fill=blue!20,draw,
 font=\sffamily\Large\bfseries,minimum size=15mm}]

 \node[state,initial,accepting] (AlexaReady) {Ready};
 \node[state] (Welcome) [below of=AlexaReady] {Load};
 \node[state] (Wait) [below of=Welcome] {Wait};
 \node[state] (set) [left of=Wait] {Set};
 \node[state] (get) [below of=Wait] {Get};


 \path[every node/.style={font=\sffamily\small,
 		fill=white,inner sep=1pt}]
 (AlexaReady) edge [bend left=20] node[right=1mm] {Nutzer: \glqq Starte die Sprachsteuerung\grqq{}} (Welcome)
 (Welcome) edge [bend left=20] node[right=1mm] {Alexa: \glqq Willkommen bei der Bachelorarbeit\grqq{}} (Wait)

 (set) edge [bend left=20] node[above=1mm] {Alexa: \glqq Okay, Licht an.\grqq{}} (Wait)
 (Wait) edge [bend left=20] node[below=1mm] {Nutzer: \glqq Licht an\grqq{}} (set)

 (get) edge [bend left=20] node[left=1mm] {Alexa: \glqq Licht, aktueller Zustand: an.\grqq{}} (Wait)
 (Wait) edge [bend left=20] node[right=1mm] {Nutzer: \glqq Ist das Licht an?\grqq{}} (get)

 (Wait) edge [bend left=20] node[left=1mm] {Nutzer: \glqq Danke\grqq{}} (AlexaReady)
 (Wait) edge [loop right] node[right=1mm] {(Ausgabe einer Hilfe)} (Wait);
 
\end{tikzpicture}
\caption{Automat des dialogbasierten Alexa Skill}
\label{fig:alexa-automat}
\end{figure}

\subsection{verlinken}

Verlinken mit Abbildungsnummer und Seite: \\
In der Abbildung \ref{fig:alexa-automat} auf Seite \pageref{fig:alexa-automat} wird gezeigt wie nice tikz sein kann.


\chapter{Listings}

Immer nice, Listings mit Synthax-Highlighting!\\

\lstinputlisting[style=custombash, firstline=31, lastline=48,
caption={Datei: setup-posix-mqtt.sh, Zeilen 31-48},
label={fig:setupmqtt}
]{sourcecode/setup-posix-mqtt.sh}

So wird verlinkt: \\

Das \autoref{fig:setupmqtt} auf Seite \pageref{fig:setupmqtt} zeigt in den Zeilen 34 bis 56 wie hässlich ein Shellscript werden kann. Hier wurde der Style custombash verwendet. Im \autoref{lst:skilllaunch} auf Seite \pageref{lst:skilllaunch} wird der customc-Style verwendet.


\lstinputlisting[style=customc,
caption={IntentHandler für den LaunchRequest},
label={lst:skilllaunch}
]{sourcecode/index-launchRequest.js}
